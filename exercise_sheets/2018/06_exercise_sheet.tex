\documentclass[a4paper,10pt]{article}
\topmargin-1cm
\addtolength{\textheight}{2.5cm}
\addtolength{\textwidth}{2cm}
\usepackage{times}

\usepackage{verbatim}
\usepackage{color}
\usepackage{listings}
\usepackage{amsmath}
\usepackage{graphicx}
\usepackage[german]{babel}
\usepackage[latin1]{inputenc}

\setlength{\parindent}{0cm}

\title{Aufgabe 6 Programmieren HF-ICT}
\author{David Herzig}
\date{November 2017}

% some listings configurations:

\lstset{numbers=left,
        numberstyle=\tiny,
        keywordstyle=\color{blue}\bfseries\sffamily,
        identifierstyle=\ttfamily,
        commentstyle=\em,
        stringstyle=\ttfamily,
        extendedchars=true,
        showstringspaces=false,
        language=c++}
%

\renewcommand{\arraystretch}{2.0}

\begin{document}

HF-ICT - H"ohere Fachschule f"ur Informations- und Kommunikationstechnologie\\
Programmieren 3. Semester, Algorithmen und Datenstrukturen\\
David Herzig

\vspace{2mm}

\begin{center}
{\Large \bf Algorithmen und Datenstrukturen}\\
Exercise sheet 6
\end{center}

\vspace{2mm}

\line(1,0){400}

\vspace{5mm}

\section{Graph - All nodes are reachable}
Gegeben ist ein gerichteter Graph in der Form wie im Unterricht besprochen:

\begin{lstlisting}
Graph g;
\end{lstlisting}

Implementieren Sie die folgende Funktion

\begin{lstlisting}
bool allNodeAreReachable(Graph *g,
                         int start);
\end{lstlisting}

Diese Funktion erh"alt als Parameter einen Graphen, die Anzahl Knoten und den
Start Knoten. Die Funktion liefert \verb|true| falls alle anderen Knoten im Graphen
von \verb|start| erreichbar sind. Falls ein oder mehrere Knoten nicht erreichbar sind,
so liefert die Funktion \verb|false|.

\section{Graph - Shortest Reach}
Gegeben ist ein ungerichteter Graph in der Form wie im Unterricht besprochen:

\begin{lstlisting}
Graph g;
\end{lstlisting}

Implementieren Sie die folgende Funktion

\begin{lstlisting}
int shortestReach(Graph *g,
                  int start, int end);
\end{lstlisting}

Diese Funktion liefert die minimale Anzahl Knoten, welche von \verb|start| nach \verb|end|
passiert werden muss. Falls \verb|start| und \verb|end| gleich sind, liefert die Funktion
\verb|0|. Falls kein Weg zwischen \verb|start| und \verb|end| existiert, so liefert die
Funktion den Wert \verb|-1|.

\section{Graph - Get Path}
Gegeben ist ein gerichteter, gewichteter Graph in der Form wie im Unterricht besprochen:

\begin{lstlisting}
Graph g;
\end{lstlisting}

Implementieren Sie die folgende Funktion

\begin{lstlisting}
vector<int> getPath(Graph *g
                    int start, int end);
\end{lstlisting}

Diese Funktion liefert einen Vector, welche alle Knoten (g"unstigster Weg) beinhaltet, welche von 
\verb|start| nach \verb|end| passiert werden m"ussen.

\end{document}

