\documentclass[a4paper,10pt]{article}
\topmargin-1cm
\addtolength{\textheight}{2.5cm}
\addtolength{\textwidth}{2cm}
\usepackage{times}

\usepackage{verbatim}
\usepackage[dvips]{graphicx}
\usepackage[german]{babel}
\usepackage[latin1]{inputenc}

\usepackage{color}
\usepackage{listings}
\definecolor{lbcolor}{rgb}{0.92,0.92,0.92}
\definecolor{cmtcolor}{rgb}{0.0,0.5,0.0}
\lstset{numbers=left,
        numberstyle=\tiny,
        identifierstyle=\ttfamily,
        commentstyle=\em,
        stringstyle=\ttfamily,
        extendedchars=true,
        showstringspaces=false,
        language=c++,
        commentstyle=\color{cmtcolor}}

\setlength{\parindent}{0cm}

\title{Latex Template}
\author{Patrik S"utterlin}
\date{WS2013}

\begin{document}

HF-ICT - H"ohere Fachschule f"ur Informations- und Kommunikationstechnologie\\
CH 4132 Muttenz\\
P.S"utterlin

\vspace{2mm}

\begin{center}
{\Large \bf Algorithmen und Datenstrukturen}\\
Exercise sheet 7
\end{center}

\vspace{2mm}

\line(1,0){400}

\vspace{5mm}

\section{Knacknuss (Kombinatorik)}

L"osen Sie folgendes mathematisches Problem: \textbf{MATHE + IST + MEIN = LEBEN} \\

Jeder Buchstabe in der Gleichung representiert eine der gelisteten Variablen: \\ \{A, B, E, H, I, L, M, N, S, T\} \\
Den einzelnen Variabeln sollen folgende Ganzzahlen zugewiesen werden: \\ \{0, 1, 2, 3, 4, 5, 6, 7, 8, 9\} \\

Die Reihenfolge der Zuweisung ist beliebig. Jede Ziffer muss in der Kombination genau einmal vorkommen. \\

Beispiel: A=9, B=8, E=1, H=2, I=3, L=4, M=5, N=6, S=7, T=0 \\

Setzt man diese Kombination nun in die Gleichung ein, so ergibt dies: \\ $\Rightarrow$ \textbf{59021 + 370 + 5136 = 41816} \\ 
$\Rightarrow$ \textbf{64527 = 41816} \\ 

Schreiben Sie ein Programm welches s"amtliche m"oglichen korrekten L"osungen zu diesem Problem auf den Bildschirm ausgibt:

{ \small
\lstinputlisting{code/07_01_main.cpp}
}

\section{Rucksackproblem}

Gegeben sind mehrere Lebensmittel mit einem bestimmten N"ahrwert. \\

Es soll nun eine Auswahl an Lebensmitteln getroffen werden, die in einen Rucksack mit einer vorgegebenen Gewichtsschranke eingepackt werden k"onnen. \\

Der N"ahrwert s"amtlicher eingepackter Lebensmittel soll maximiert werden so dass der Rucksacktr"ager sich m"oglichst lange von den Lebensmitteln ern"ahren kann. \\

Beispiel:

\begin{itemize}
	\item 12 kg Sellerie $\Rightarrow$ N"ahrwert = 4
	\item 4  kg Bonbons $\Rightarrow$ N"ahrwert = 10
	\item 2  kg Kn"ackebrot $\Rightarrow$ N"ahrwert = 2
	\item 1  kg "Apfel $\Rightarrow$ N"ahrwert = 2
	\item 1  kg Karotten $\Rightarrow$ N"ahrwert = 1
\end{itemize}

Bei einer zul"assigen Maximallast von 15kg im Rucksack m"ussten Sie einpacken: \textbf{Bonbons, Kn"ackebrot, "Apfel, Karotten} (Gesamtn"ahrwert = 15 / Tragelast = 8kg). \\

Schreiben Sie ein Programm welches einen Vektor mit der optimalen Packliste der Lebensmittel zur"uckgibt: 

{ \small
\lstinputlisting{code/07_02_main.cpp}
}

\section{Pr"ufung: Graph ist ein/kein Baum}

Schreiben Sie ein Funktion in der Klasse GraphUtil welche ''true'' zur"uckliefert, falls ein ungerichteter Graph ein Baum darstellt - ansonsten geben Sie ''false'' zur"uck:

{ \small
\lstinputlisting{code/07_03_main.cpp}
}

\textbf{Hinweis:} Im Gegensatz zum bin"aren Baum darf dieser Baum pro Parent Element null bis \textit{n} Child Elemente besitzen.

\section{Maximales Produkt Sub-Array Problem}

Suchen Sie in einem Array (vector$<$int$>$) jeweils das zusammenh"angende Sub-Array welches das gr"osst m"ogliche errechenbare Produkt (d.h. die Multiplikation s"amtlicher enthaltenen Elementwerte) aufweist. Geben Sie das gr"osst m"ogliche Produkt zur"uck.\\

\textbf{Beispiel:} \{-2, -3, 0, -2, -40\} \\

Das enthaltene Sub-Array mit dem gr"ossten Produkt w"are \{-2, -40\}. \\ Der R"uckgabewert der Methode w"are somit 80. \\

Schreiben Sie das passende Programm f"ur diese Aufgabe:

{ \small
\lstinputlisting{code/07_04_main.cpp}
}


\end{document}

