\documentclass[a4paper,10pt]{article}
\topmargin-1cm
\addtolength{\textheight}{2.5cm}
\addtolength{\textwidth}{2cm}
\usepackage{times}

\usepackage{verbatim}
\usepackage{color}
\usepackage{listings}
\usepackage[dvips]{graphicx}
\usepackage[german]{babel}
\usepackage[latin1]{inputenc}

\setlength{\parindent}{0cm}

\title{Aufgabe 3 Programmieren HF-ICT}
\author{David Herzig}
\date{August 2015}

% some listings configurations:

\lstset{numbers=left,
        numberstyle=\tiny,
        keywordstyle=\color{blue}\bfseries\sffamily,
        identifierstyle=\ttfamily,
        commentstyle=\em,
        stringstyle=\ttfamily,
        extendedchars=true,
        showstringspaces=false,
        language=c++}
%

\renewcommand{\arraystretch}{2.0}

\begin{document}

HF-ICT - H"ohere Fachschule f"ur Informations- und Kommunikationstechnologie\\
Programmieren 3. Semester, Algorithmen und Datenstrukturen\\
David Herzig

\vspace{2mm}

\begin{center}
{\Large \bf Algorithmen und Datenstrukturen}\\
Exercise sheet 3
\end{center}

\vspace{2mm}

\line(1,0){400}

\vspace{5mm}

\section{M"unzkombinationen}
In dieser Aufgabe soll ein Algorithmus implementiert werden, welcher
f"ur einen bestimmten Geldbetrag die minimale Anzahl M"unzen berechnet.

\vspace{3mm}

Der Algorithmus erh"alt als Input den Betrag sowie ein Array mit den 
verf"ugbaren M"unzen.

\begin{lstlisting}
class CoinCombination {
public:
  static void printCoinCombination(
    double value, // Betrag
    const double *coins, // Pointer auf Muenzarray
    const int ARRAY_SIZE); // Groesse des Muenzarray
};
\end{lstlisting}

Beispiel:\\
Betrag: 23 CHF\\
M"unzen: {0.05, 0.10, 0.20, 0.50, 1, 2, 5}\\
Resultat: 5, 5, 5, 5, 2, 1 (minimale Anzahl M"unzen) 

\section{Palindrome}
In dieser Aufgabe soll eine Methode implementiert werden, welche mit Hilfe von Rekursion pr"uft,
ob eine Zeichenkette ein Palindrome ist.

\vspace{3mm}

\begin{lstlisting}
class StringUtil {
public:
  static bool isPalindrome(string input);
};
\end{lstlisting}

\section{L"angster gemeinsamer Substring}
In dieser Aufgabe soll eine Methode implementiert werden, welche von 2 Zeichenketten den l"angsten
Substring findet, welcher in beiden Zeichenketten vorkommt.

\vspace{3mm}

\begin{lstlisting}
class StringUtil {
public:
  static string lcs(string input1, string input2);
};
\end{lstlisting}


\end{document}

