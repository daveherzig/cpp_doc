\documentclass[a4paper,10pt]{article}
\topmargin-1cm
\addtolength{\textheight}{2.5cm}
\addtolength{\textwidth}{2cm}
\usepackage{times}

\usepackage{verbatim}
\usepackage{color}
\usepackage{listings}
\usepackage[dvips]{graphicx}
\usepackage[german]{babel}
\usepackage[latin1]{inputenc}

\setlength{\parindent}{0cm}

\title{Aufgabe 8 Programmieren HF-ICT}
\author{David Herzig}
\date{August 2015}

% some listings configurations:

\lstset{numbers=left,
        numberstyle=\tiny,
        keywordstyle=\color{blue}\bfseries\sffamily,
        identifierstyle=\ttfamily,
        commentstyle=\em,
        stringstyle=\ttfamily,
        extendedchars=true,
        showstringspaces=false,
        language=c++}
%

\renewcommand{\arraystretch}{2.0}

\begin{document}

HF-ICT - H"ohere Fachschule f"ur Informations- und Kommunikationstechnologie\\
Programmieren 3. Semester, Algorithmen und Datenstrukturen\\
David Herzig

\vspace{2mm}

\begin{center}
{\Large \bf Algorithmen und Datenstrukturen}\\
Exercise sheet 8
\end{center}

\vspace{2mm}

\line(1,0){400}

\vspace{5mm}

\section{Quadrat Finder}
Gegeben ist eine Matrix der Gr"osse $m \cdot n$. Jeder Feld dieser Matrix ist
entweder \verb|true| oder \verb|false|. Die Aufgabe besteht darin, herauszufinden,
ob alle \verb|true| Felder in einem Quadrat sind.

\vspace{3mm}

\begin{lstlisting}
class Matrix {
public:
  bool getCellState(int m, int n);
  int getNumberOfRows();
  int getNumberOfColumns();
};

bool checkMatrix(const Matrix & matrix) {
  // returns true if all true fields are in a square
}
\end{lstlisting}

\section{Zwischensumme}
Gegeben ist ein Integer Array der Gr"osse N. Weiterhin ist eine Zahl K gegeben.
Implementieren Sie eine Funktion, welche als Parameter den Array sowie die Zahl K
bekommt. Diese Methode gibt \verb|true| zur"uck, falls die Zahl K als Summe
von Elementen aus dem Array gebildet werden kann (jedes Element darf maximal
einmal verwendet werden). Falls nicht,gibt die Methode \verb|false|
zur"uck.

\vspace{3mm}

\begin{lstlisting}
bool subsetsum(int *array, int size, int k) {
  // your implementation
}
\end{lstlisting}

\end{document}

