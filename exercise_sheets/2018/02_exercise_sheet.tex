\documentclass[a4paper,10pt]{article}
\topmargin-1cm
\addtolength{\textheight}{2.5cm}
\addtolength{\textwidth}{2cm}
\usepackage{times}

\usepackage{verbatim}
\usepackage{color}
\usepackage{listings}
\usepackage[dvips]{graphicx}
\usepackage[german]{babel}
\usepackage[latin1]{inputenc}

\setlength{\parindent}{0cm}

\title{Aufgabe 2 Programmieren HF-ICT}
\author{David Herzig}
\date{August 2015}

% some listings configurations:

\lstset{numbers=left,
        numberstyle=\tiny,
        keywordstyle=\color{blue}\bfseries\sffamily,
        identifierstyle=\ttfamily,
        commentstyle=\em,
        stringstyle=\ttfamily,
        extendedchars=true,
        showstringspaces=false,
        language=c++}
%

\renewcommand{\arraystretch}{2.0}

\begin{document}

HF-ICT - H"ohere Fachschule f"ur Informations- und Kommunikationstechnologie\\
Programmieren 3. Semester, Algorithmen und Datenstrukturen\\
David Herzig

\vspace{2mm}

\begin{center}
{\Large \bf Algorithmen und Datenstrukturen}\\
Exercise sheet 2
\end{center}

\vspace{2mm}

\line(1,0){400}

\vspace{5mm}

\section{Aktienkurse}
In dieser Aufgabe soll eine Methode implementiert werden, welche als Parameter einen Array enth"alt. Dieser
Array beinhaltet die Aktienpreise eines bestimmten Unternehmen f"ur einen Tag. Der Arrayindex ist dabei
die Minute seit Start (B"orse "offnet um 09:00 und schliesst um 17:30).

\vspace{3mm}

Beispiele:\\
\verb|stockprices[0] = 100 // Wert der Aktie um 09:00|\\
\verb|stockprices[1] = 101 // Wert der Aktie um 09:01|\\
\verb|stockprices[10] = 89 // Wert der Aktie um 09:10|\\
\verb|stockprices[70] = 75 // Wert der Aktie um 10:10|\\
\verb|stockprices[440] = 110 // Wert der Aktie um 16:20|

\vspace{3mm}

Finden Sie den maximalem Gewinn der an diesem Tag erzielt werden konnte.\\
{\bf ACHTUNG}: \verb|"|shorting\verb|"| (Verkauf bevor Kauf) ist nicht erlaubt. Ebenso ist es nicht erlaubt
die Aktien zum gleichen Zeitpunkt zu kaufen und zu verkaufen.
Es kann durchaus m"oglich sein, dass der Gewinn negativ ist.

\vspace{3mm}

\begin{lstlisting}
class Trader {
public:
  static int calculateMaxProfit(const int *values, const int ARRAY_SIZE);
};
\end{lstlisting}

\section{Gr"osstes Produkt}
In dieser Aufgabe soll eine Methode implementiert werden, welche als Parameter einen Array mit Integer Werten
enth"alt. Aus diesem Array soll nun das gr"osste Produkt gefunden werden, welches mit 3 Elementen aus diesem
Array erstellt werden kann.

\vspace{3mm}

\begin{lstlisting}
class ArrayUtil {
public:
  static int highestProduct(int *values, const int ARRAY_SIZE);
};
\end{lstlisting}

\section{Anagramm Check}
In dieser Aufgabe soll eine Methode implementiert werden, welche als Parameter zwei Strings enth"alt.
Die Methode liefert den Wert \verb|true|, falls diese beiden Strings Anagramme sind. Anonsten \verb|false|.

\begin{lstlisting}
class StringUtil {
public:
  static bool anagramCheck(string s1, string s2);
};
\end{lstlisting}

\end{document}

