\documentclass[a4paper,10pt]{article}
\topmargin-1cm
\addtolength{\textheight}{2.5cm}
\addtolength{\textwidth}{2cm}
\usepackage{times}

\usepackage{verbatim}
\usepackage{color}
\usepackage{listings}
\usepackage[dvips]{graphicx}
\usepackage[german]{babel}
\usepackage[latin1]{inputenc}

\setlength{\parindent}{0cm}

\title{Aufgabe 1 Programmieren HF-ICT}
\author{David Herzig}
\date{August 2015}

% some listings configurations:

\lstset{numbers=left,
        numberstyle=\tiny,
        keywordstyle=\color{blue}\bfseries\sffamily,
        identifierstyle=\ttfamily,
        commentstyle=\em,
        stringstyle=\ttfamily,
        extendedchars=true,
        showstringspaces=false,
        language=c++}
%

\renewcommand{\arraystretch}{2.0}

\begin{document}

HF-ICT - H"ohere Fachschule f"ur Informations- und Kommunikationstechnologie\\
Programmieren 3. Semester, Algorithmen und Datenstrukturen\\
David Herzig

\vspace{2mm}

\begin{center}
{\Large \bf Algorithmen und Datenstrukturen}\\
Exercise sheet 1
\end{center}

\vspace{2mm}

\line(1,0){400}

\vspace{5mm}

\section{Analyse}
Welche Laufzeit hat der folgende Code?

\begin{lstlisting}
void foo(int *array, int size) {
  int sum = 0;
  int product = 1;
  for (int i=0; i<size; i++) {
    sum += array[i];
  }
  for (int i=0; i<size; i++) {
    product *= array[i];
  }
  cout << sum << "," << product << endl;
}
\end{lstlisting}

alc
\section{Array Multiplikation}
In dieser Aufgabe soll eine Methode implementiert werden, welche als Parameter einen Array enth"alt.
Das Resultat ist ein Array der gleichen Gr"osse. Jedes Element dieses Arrays beinhaltet die
Multiplikation aller Elemente (ausser sich selbst) des Input Arrays.

\vspace{3mm}

Beispiel:\\
Input Array: \verb|{12, 4, 7, 3}|\\
Output Array: \verb|{4*7*3, 12*7*3, 12*4*3, 12*4*7} = {84, 252, 144, 336}|

\vspace{3mm}

{\bf ACHTUNG}: Der Divisionsoperator darf in dieser Aufgabe NICHT verwendet werden!

\vspace{3mm}

\begin{lstlisting}
class ArrayUtil {
public:
  static long* multiplyArrayValues(const long *values, const int ARRAY_SIZE);
};
\end{lstlisting}

\section{String Check}
In dieser Aufgabe soll eine Methode implementiert werden, welche als Parameter einen String enth"alt.
Das Resultat ist ein Boolean. \verb|true| falls alle Buchstaben unique sind (alle Buchstaben kommen nur
einmal vor), sonst \verb|false|.

\vspace{3mm}

\begin{lstlisting}
class StringUtil {
public:
  static bool checkUniqueness(string s);
};
\end{lstlisting}

\end{document}

