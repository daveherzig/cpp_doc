\documentclass[a4paper,10pt]{article}
\topmargin-1cm
\addtolength{\textheight}{2.5cm}
\addtolength{\textwidth}{2cm}
\usepackage{times}

\usepackage{verbatim}
\usepackage{color}
\usepackage{listings}
\usepackage[dvips]{graphicx}
\usepackage[german]{babel}
\usepackage[latin1]{inputenc}

\setlength{\parindent}{0cm}

\title{Aufgabe 4 Programmieren HF-ICT}
\author{David Herzig}
\date{September 2017}

% some listings configurations:

\lstset{numbers=left,
        numberstyle=\tiny,
        keywordstyle=\color{blue}\bfseries\sffamily,
        identifierstyle=\ttfamily,
        commentstyle=\em,
        stringstyle=\ttfamily,
        extendedchars=true,
        showstringspaces=false,
        language=c++}
%

\renewcommand{\arraystretch}{2.0}

\begin{document}

HF-ICT - H"ohere Fachschule f"ur Informations- und Kommunikationstechnologie\\
Programmieren 3. Semester, Algorithmen und Datenstrukturen\\
David Herzig

\vspace{2mm}

\begin{center}
{\Large \bf Algorithmen und Datenstrukturen}\\
Exercise sheet 4
\end{center}

\vspace{2mm}

\line(1,0){400}

\vspace{5mm}

\section{Lonely Element}
Implementieren Sie eine Methode \verb|getLonelyElement| welche als
Parameter einen \verb|string| bekommt. Jeder Character in diesem
string kommt zweimal vor, ausser eines. Die Methode soll das
Element zur"uckliefern, welches nur einmal vorkommt.

\begin{lstlisting}
class StringUtil {
public:
  static char getLonelyElement(string input);
};
\end{lstlisting}

Beispiel:\\
Input: abcdedcba11\\
Resultat: e (das Element e kommt nur einmal vor)

\vspace{3mm}

{\bf ACHTUNG: Die Implementierung muss besser sein als $O(n^2)$}

\section{Plus Minus}
Implementieren Sie eine Methode \verb|analyseArray| welche als Parameter
einen \verb|vector| mit Integer Werten bekommt. Diese Werte sind entweder
Positiv, Negativ oder Null. Die Methode soll nun berechnen, wie viel Prozent
der Werte Positiv, Negativ resp. Null sind. Diese drei Werte werden als
Objekt der Klasse \verb|Result| zur"uckgeliefert.

\begin{lstlisting}
class Result {
public:
  int nPositive;
  int nZero;
  int nNegative;
};

class ArrayUtil {
public:
  static Result analyseArray(vector<int> input);
};
\end{lstlisting}

Beispiel:\\
Input: {1, 10, 0, -6, -9, 2, 5}\\
L"osung: \verb|[nPositive: 57.1%; nZero: 14.3%; nNegative: 28.6%]|

\end{document}

