\section{STL - Standard Template Library}

\begin{frame}[fragile]
  \frametitle{STL - Standard Template Library}
  \begin{itemize}
  \item Containers\\
  {\small STL contains sequence containers, associative containers and container adaptors}
  \item Iterators\\
  {\small STL contains input iterators, output iterators, forward iterators, bidirectional iterators and random access iterators}
  \item Algorithms\\
  {\small STL contains a large number of algorithms (e.g. sorting, searching)}
  \end{itemize}
\end{frame}

\begin{frame}[fragile]
  \frametitle{STL - Containers}
  Sequence Container
  {\small
  \begin{itemize}
  \item vector
  \item deque
  \item list
  \item array (C++11)
  \end{itemize}
  Adaptive Containers
  \begin{itemize}
  \item stack
  \item queue
  \end{itemize}
  Associative Containers (elements are in order)
  \begin{itemize}
  \item map, multimap
  \item set, multiset
  \end{itemize}
  \verb|http://www.cplusplus.com/reference/|
  }
\end{frame}

\begin{frame}[fragile]
\frametitle{Vector, Deque, List}
{\tiny
\begin{lstlisting}
int main(int argc, char **argv) {

  deque<int> c1;
  vector<int> c2;
  list<int> c3;

  // insert at the end
  c1.push_back(rand());
  c2.push_back(rand());
  c3.push_back(rand());

  // insert at the start
  c1.push_front(rand());
  // c2.push_front not possible
  c3.push_front(rand());
\end{lstlisting}
}
\end{frame}

\begin{frame}[fragile]
\frametitle{Vector, Deque, List}
{\tiny
\begin{lstlisting}
  // insert at a specific position
  list<int>::iterator iter3 = c3.begin();
  iter3++;
  c3.insert(iter3, rand());

  // delete at a specific position
  vector<int>::iterator iter2 = c2.begin();
  c2.erase(iter2);

  // loop through elements
  deque<int>::iterator iter1;
  for (iter1=c1.begin(); iter1!=c1.end(); iter1++) {
    cout << *iter1 << endl;
  }

  return 0;
}
\end{lstlisting}
}
\end{frame}

\begin{frame}[fragile]
  \frametitle{Performance Comparison}
  {\tiny
  Deque
  \begin{verbatim}
Time (push_back, 10 Mio Elements): 0.123527
Time (insert, 1000 Elements): 11.0062
Time (erase, 1000 Elements): 10.9873
  \end{verbatim}
  Vector
  \begin{verbatim}
Time (push_back, 10 Mio Elements): 0.166174
Time (insert, 1000 Elements): 2.12804
Time (erase, 1000 Elements): 2.09296
  \end{verbatim}
  List
  \begin{verbatim}
Time (push_back, 10 Mio Elements): 0.806527
Time (insert, 1000 Elements): 8.5e-05
Time (erase, 1000 Elements): 6.7e-05
  \end{verbatim}
  }
\end{frame}

\begin{frame}[fragile]
\frametitle{Map}
{\tiny
\begin{lstlisting}
map<int, string> employeeMap;

// insert
employeeMap[0] = "HERZIG";
employeeMap[1] = "BALE";
employeeMap[0] = "WAYNE"; // old value will be overwritten

pair<map<int, string>::iterator, bool> ret;
ret = employeeMap.insert(pair<int, string>(2, "PARKER"));
// ret.second is true
ret = employeeMap.insert(pair<int, string>(2, "CLARK"));
// ret.second is false, because key 2 already exists

// loop through elements
map<int, string>::iterator iter;
for (iter=employeeMap.begin(); iter!=employeeMap.end(); iter++) {
  int key = iter->first;
  string value = iter->second;
}
\end{lstlisting}
}
\end{frame}

\begin{frame}[fragile]
\frametitle{Set}
{\tiny
\begin{lstlisting}
set<int> data;

pair<set<int>::iterator,bool> ret;
for (int i=0; i<100; i++) {
  int number = rand() % 100;
  ret = data.insert(number);
  if (!(ret.second)) {
    cout << "Could not insert " << number << endl;
  }
}
cout << data.size() << endl;

set<int>::iterator it;
for (it=data.begin(); it!=data.end(); it++) {
  cout << *it << endl;
}
\end{lstlisting}
}
\end{frame}

\begin{frame}[fragile]
\frametitle{STL - Algorihtms}
The algorithms library defines functions for a variety of purposes (e.g. searching, sorting,
counting, manipulating) that operate on ranges of elements. Note that a range is defined as
\verb|[first, last)| where last refers to the element past the last element to inspect or modify.
\begin{itemize}
\item Non-modifying sequence operations
\item Modifying sequence operations
\end{itemize}

\end{frame}

\begin{frame}[fragile]
\frametitle{count}
{\tiny
Sucht nach einem bestimmten Element in einem
Container und gibt die Anzahl der Vorkommnisse zurück.

\lstinputlisting{code/stl/count.cpp}

\emph{Ähnliche Algorithmen}: \verb|count_if|
}
\end{frame}

\begin{frame}[fragile]
\frametitle{all\_of}
{\tiny
Prüft ob eine Bedinung für alle Elemente wahr ist.\\
\emph{ACHTUNG:} Teil von \verb|C++0x|, (\verb|g++ -std=c++0x file.cpp|).

\lstinputlisting{code/stl/allof.cpp}

\emph{Ähnliche Algorithmen}: \verb|any_of, none_of|
}
\end{frame}

\begin{frame}[fragile]
\frametitle{find}
{\tiny
Sucht nach einem Element im Container und gibt den Iterator auf dieses Element zurück.
Kommt das Element im Container nicht vor, so ist der Rückgabewert gleich \verb|container.end()|.

\lstinputlisting{code/stl/find.cpp}

\emph{Ähnliche Algorithmen}: \verb|find_if, find_if_not, find_end, find_first_of|
}
\end{frame}

\begin{frame}[fragile]
\frametitle{is\_permutation}
{\tiny
Prüft zwei Container ob die gleichen Elemente vorkommen. Egal in welcher Reihenfolge.\\
\emph{ACHTUNG:} Teil von \verb|C++0x|, (\verb|g++ -std=c++0x file.cpp|).

\lstinputlisting{code/stl/permutation.cpp}
}
\end{frame}

\begin{frame}[fragile]
\frametitle{copy}
{\tiny
Kopiert die Elemente aus einem Container in einen anderen. Der Ziel Container muss
dafür entsprechend Platz zur Verfügung stellen.

\lstinputlisting{code/stl/copy.cpp}
}
\end{frame}

\begin{frame}[fragile]
\frametitle{move}
{\tiny
\emph{ACHTUNG:} Teil von \verb|C++0x|, (\verb|g++ -std=c++0x file.cpp|).

\lstinputlisting{code/stl/move.cpp}
}
\end{frame}

\begin{frame}[fragile]
\frametitle{replace}
{\tiny
Ein Element durch ein anderes Element ersetzen.

\lstinputlisting{code/stl/replace.cpp}
}
\end{frame}

\begin{frame}[fragile]
\frametitle{fill}
{\tiny
Fügt allen Elementen eines Containers einen bestimmten Wert zu.

\lstinputlisting{code/stl/fill.cpp}
}
\end{frame}

\begin{frame}[fragile]
\frametitle{unique}
{\tiny
Entfernt alle aufeinanderfolgenden Duplikate (ausser das erst gefundene) aus einem Container.
Die Grösse muss nach dem Entfernen manuell angepasst werden. Dazu kann der Rückgabewert
verwendet werden. Dies ist ein Iterator auf das letzte gültige Element.

\lstinputlisting{code/stl/unique.cpp}
}
\end{frame}

\begin{frame}[fragile]
\frametitle{shuffle}
{\tiny
Durchmischt die Elemente eines Containers.
\emph{ACHTUNG:} Teil von \verb|C++0x|, (\verb|g++ -std=c++0x file.cpp|).

\lstinputlisting{code/stl/shuffle.cpp}
}
\end{frame}

\begin{frame}[fragile]
\frametitle{min\_element}
{\tiny
Sucht nach dem Minimum in einem Container.

\lstinputlisting{code/stl/minelement.cpp}
}
\end{frame}

\begin{frame}[fragile]
\frametitle{min, max}
{\tiny
Template Funktion um den kleineren resp. grösseren Wert zweier Parameter
zu bestimmen.

\lstinputlisting{code/stl/min.cpp}
}
\end{frame}

\begin{frame}[fragile]
\frametitle{sort}
{\tiny
Sortiert alle Elemente in einem Container.

\lstinputlisting{code/stl/sort.cpp}
}
\end{frame}

\begin{frame}[fragile]
\frametitle{Exercise}
{\tiny
\begin{exercise}
Insert 10000 random numbers (0..10000) into a vector. Write a program which counts
the number of different numbers (e.g. 1, 2, 1, 3, 2: contains 3 elements)
\end{exercise}
\begin{exercise}
Insert 10000 random numbers (0..10000) into a vector. Remove all even numbers.
\end{exercise}
\begin{exercise}
Insert 100 numbers (in order: 0,1,2,...) into a vector. Shuffle that vector.
\end{exercise}
\begin{exercise}
Insert 500 random numbers (0..1000) into a vector v1 and into a vector v2. Print
out all numbers which are only in v1 and v2.
\end{exercise}
}
\end{frame}
