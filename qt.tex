\section{Programming with QT}

\frame
{
\frametitle{What is QT?}
\begin{itemize}
\item Company History: Trolltech (1994-2008), Nokia (2008-2012), Digia (2012-2014), The QT Company (2014-)
\item QT is a cross-platform application and UI framework
\item QT provides intuitive C++ class libraries
\item Current Version: 5.10.0 (7-DEC-2017)
{\tiny
\begin{itemize}
\item 3D Graphics with OpenGL
\item Multithreading
\item Network Connectivity
\item ...
\end{itemize}
}
\item QT provides portability across several platforms\\
\begin{itemize}
\item Windows
\item OSX
\item Unix/Linux
\item Android
\item ...
\end{itemize}
\item QT provides development tools (IDE)
\end{itemize}
}

\subsection{QT Development Environment}
\frame
{
\frametitle{QT Development Environment}
Requirements
\begin{itemize}
\item MinGW
\item QT SDK
\item Editor: Notepad++, emacs, Ultra Edit, Atom, ...
\end{itemize}
\vspace{5mm}
Development Environment
\begin{itemize}
\item Console
\item QT Creator
\item Visual Studio, Code Blocks, ... (not supported@hf-ict)
\end{itemize}
}

\frame
{
\frametitle{Development with Console}
\begin{itemize}
\item Change to the directory containing the source code
\item Type \emph{qmake -project [CONFIG+=console]}
\item Type \emph{qmake}
\item Type \emph{mingw32-make}
\end{itemize}

Ensure that the correct Environment variables are set (PATH)!\\
\begin{itemize}
\item QT bin folder
\item MINGW bin folder
\end{itemize}
}

\frame
{
\frametitle{Development with QT Creator}
\begin{itemize}
\item Start the QT Creator application
\item File - New - QT 5 Gui Application
\item Enter project name
\item Select required modules
\item Start the application with Build - Run
\end{itemize}
}

\frame
{
\frametitle{Development with QT Creator}
\includegraphics[width=250pt]{img/qtcreator.png}
}

\frame
{
\frametitle{Resources}
The complete Documentation, API and Examples can be found at
\begin{itemize}
\item https://doc.qt.io
\end{itemize}
Books:
\begin{itemize}
\item Mastering Qt 5\\
Guillaume Lazar, Robin Penea\\
\includegraphics[width=90pt]{img/qt5book.jpg}
\end{itemize}
}

\frame
{
\frametitle{QT - HF-ICT}
\begin{itemize}
\item Introduction
\item Small Projects (QT Basics, Layouts, Slots, ...)
\item Fractals (QT Painting)
\item Game (Multithreading)
\end{itemize}
}

\subsection{Hello World - Hello QT - Basics}
\frame
{
\frametitle{Hello QT - My first QT program}
\lstinputlisting{code/qt/helloqt/helloqt.cpp}
\includegraphics[width=85pt]{img/helloqt.png}
}

\frame
{
\frametitle{Exercise - Hello QT}
\begin{exercise}
\begin{itemize}
\item Setup your QT development environment
\item Write a \emph{Hello QT} application
\end{itemize}
\includegraphics[width=85pt]{img/helloqt.png}
\end{exercise}
}

\begin{frame}[fragile]
\frametitle{QWidget}
Every gui element (single element or a group of element) in QT is a Widget (Window Gadget).
\vspace{5mm}
\includegraphics[width=280pt]{img/widget.png}
\end{frame}

\begin{frame}[fragile]
\frametitle{Exercise - QT Standard Widgets}
\begin{exercise}
Create the following three QT applications:\\
\vspace{5mm}
\\
\includegraphics[width=75pt]{code/qt/stdwidgets/w1.png}
\hspace{5mm}
\includegraphics[width=50pt]{code/qt/stdwidgets/w2.png}
\hspace{5mm}
\includegraphics[width=100pt]{code/qt/stdwidgets/w3.png}
\end{exercise}
\end{frame}

\begin{frame}[fragile]
\frametitle{Exercise - 2 Widgets in one Window}
\begin{exercise}
Can you display two widgets in one window?\\
\vspace{5mm}
\\
\includegraphics[width=120pt]{code/qt/stdwidgets/w4.png}
\end{exercise}
\end{frame}


\begin{frame}[fragile]
\frametitle{QDebug vs cout}
\begin{itemize}
\item qDebug()
\item qInfo()
\item qWarning()
\item qCritical()
\item qFatal()
\end{itemize}

\begin{lstlisting}
qDebug() << "some debug output...";
\end{lstlisting}

Switch off debug output:
\begin{lstlisting}
qmake -project DEFINES+=QT_NO_DEBUG_OUTPUT
\end{lstlisting}

\end{frame}


\subsection{Layout Manager}
\frame
{
\frametitle{Layout Manager}
The following Layout Managers are available in QT:
\begin{itemize}
\item QHBoxLayout
\item QVBoxLayout
\item QGridLayout
\item QFormLayout
\item Null Layout
\item ...
\end{itemize}
}

\begin{frame}[fragile]
\frametitle{QHBoxLayout (QVBoxLayout)}
Horizontal arrangement of all widgets.\\
\includegraphics[width=112pt]{code/qt/hbox/hboxlayout.jpg}\\
{\tiny
\lstinputlisting{code/qt/hbox/hbox.cpp}
}
\end{frame}

\begin{frame}[fragile]
\frametitle{QHBoxLayout (QVBoxLayout)}
\includegraphics[width=100pt]{code/qt/stretch/stretch1.png}\\
\includegraphics[width=150pt]{code/qt/stretch/stretch2.png}\\
\begin{lstlisting}
button->sizePolicy().expandingDirections();
// QFlags()
input->sizePolicy().expandingDirections();
// QFlags(0x1)
// 0x1: Qt::Horizontal
\end{lstlisting}
\end{frame}

\frame
{
\frametitle{QGridLayout}
Arrangement of all widgets in a grid.\\
\includegraphics[width=100pt]{code/qt/grid/gridlayout.jpg}\\
{\tiny
\lstinputlisting{code/qt/grid/gridlayout.cpp}
}
}

\frame
{
\frametitle{QFormLayout}
The QFormLayout class manages forms of input widgets and their associated labels.\\
\includegraphics[width=100pt]{img/formlayout.png}\\
{\tiny
\lstinputlisting{code/qt/formlayout/mywidget.cpp}
}
}

\frame
{
\frametitle{Null Layout}
No layout manager\\
\includegraphics[width=72pt]{code/qt/null/xylayout.png}\\
{\tiny
\lstinputlisting{code/qt/null/mywidget.cpp}
}
}

\begin{frame}[fragile]
\frametitle{Modular Programming}
\includegraphics[width=140pt]{img/simple.png}
\hspace{4mm}
\includegraphics[width=100pt]{img/simplecd.png}
\end{frame}

\begin{frame}[fragile]
\frametitle{Modular Programming}
{\tiny
\lstinputlisting{code/qt/simple/mywidget.h}
}
\end{frame}

\begin{frame}[fragile]
\frametitle{Modular Programming}
{\tiny
\lstinputlisting{code/qt/simple/mywidget.cpp}
}
\end{frame}

\begin{frame}[fragile]
	\frametitle{Modular Programming}
	{\tiny
	\lstinputlisting{code/qt/simple/main.cpp}
	}
\end{frame}

\begin{frame}[fragile]
	\frametitle{Exercise - Modular Programming}
	\begin{exercise}
  Implement a gui to validate a credit card number (only the number).\\
  \includegraphics[width=140pt]{img/creditcard.png}\\
  The Luhn formula can be used to validate the number. But this will come in
  a later exercise!
	\end{exercise}
\end{frame}

\frame
{
	\frametitle{QMainWindow}
	A main window provides a framework for building an application's user interface.
	Qt has QMainWindow and its related classes for main window management.
	QMainWindow has its own layout to which you can add QToolBars, QDockWidgets,
	a QMenuBar, and a QStatusBar. The layout has a center area that can be occupied
	by any kind of widget.\\
	\includegraphics[width=140pt]{img/mainwindowlayout.png}

}

\frame
{
	\frametitle{Exercise 2 - MainWindow}
	Create an application according to the following class diagram and graphical user interface.\\
	\includegraphics[width=280pt]{img/helloqtcd.png}\\
	\includegraphics[width=85pt]{img/helloqt.png}
}

\frame
{
	\frametitle{Change Background Color of QLineEdit}
	\lstinputlisting{code/qt/color.cpp}
}

\frame
{
	\frametitle{Validate Input}
	There are two ways on how to validate the input values.
	\begin{itemize}
	\item Using a validator
	{\tiny
	\lstinputlisting{code/qt/validator/method1.cpp}
	}
	\item Check the values after commit
	{\tiny
	\lstinputlisting{code/qt/validator/method2.cpp}
	}
	\end{itemize}
}


\subsection{Event Handling}
\frame
{
	\frametitle{Slots}
	Every widget may fire events (Signals). These events can be catched be a method
	called slot.\\
	Take a look on the API to check which Signal and Slots are available for each
	widget.\\
	Example: Signals of the class QPushButton:
	\begin{itemize}
	\item clicked(bool checked=false)
	\item pressed()
	\item released()
	\item toggled(bool checked)
	\end{itemize}
}

\frame
{
	\frametitle{Slots}
	A class which implements a slot must
	\begin{itemize}
	\item be inherited from the class QObject (or a subclass)
	\item contain the Q\textunderscore OBJECT statement
	\item contain the public slots section
	\end{itemize}
	\lstinputlisting{code/qt/slot.cpp}
}

\frame
{
	\frametitle{Slots}
	Within the code, the Signal and Slot must be connected.
	\lstinputlisting{code/qt/connection.cpp}
	The senderObject fires the event (signal). The receiverObject contains the slot.
}

\begin{frame}[fragile]
	\frametitle{My first slot}
	\includegraphics[width=140pt]{img/simple.png}
	\hspace{4mm}
	\includegraphics[width=100pt]{img/simplecd.png}\\
	\vspace{3mm}
	Add a simple slot called \verb|onButtonClicked| in the class \verb|MyWidget|.
\end{frame}

\begin{frame}[fragile]
	\frametitle{My first slot}
	\verb|mywidget.h|
	\begin{lstlisting}
class MyWidget : public QWidget {
Q_OBJECT
private:
  ... // see slide before
public:
  ... // see slide before
public slots:
  void onButtonClicked();
	\end{lstlisting}
\end{frame}

\begin{frame}[fragile]
	\frametitle{My first slot}
	\verb|mywidget.cpp|
	\begin{lstlisting}
MyWidget::MyWidget() {
  ... // see slide before
  QOBject::connect(
    okButton, SIGNAL(clicked()),
    this, SLOT(onButtonClicked()));
}

void MyWidget::onButtonClicked() {
  // delete the content of the QLineEdit component
  inputLineEdit->clear();
}
	\end{lstlisting}
\end{frame}

\begin{frame}[fragile]
	\frametitle{Exercise 4}
	Take the code from the example above. Add a new class called \verb|EventHandler|
	and move the slot \verb|onButtonClicked| from the class \verb|MyWidget| to the
	class \verb|EventHandler|. The functionality should still be the same.
\end{frame}

\begin{frame}[fragile]
	\frametitle{Exercise 4 - Difficulties}
	The major difficulties in this exercise are:
	\begin{itemize}
	\item A new object of the class \verb|EventHandler| must be created and used
	within the \verb|QObject::connect| statement.
	\item A way must be found to access the \verb|inputLineEdit| component, which
	is now in another class than the slot.
	\end{itemize}
\end{frame}

\begin{frame}[fragile]
\end{frame}
\subsection{Painting}
\begin{frame}[fragile]
\end{frame}
\subsection{Threads}
\begin{frame}[fragile]
\end{frame}
\subsection{Database Access}
\begin{frame}[fragile]
\end{frame}
\subsection{Network Connections}
\begin{frame}[fragile]
\end{frame}
